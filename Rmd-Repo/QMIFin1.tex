% Options for packages loaded elsewhere
\PassOptionsToPackage{unicode}{hyperref}
\PassOptionsToPackage{hyphens}{url}
%
\documentclass[
]{article}
\usepackage{lmodern}
\usepackage{amssymb,amsmath}
\usepackage{ifxetex,ifluatex}
\ifnum 0\ifxetex 1\fi\ifluatex 1\fi=0 % if pdftex
  \usepackage[T1]{fontenc}
  \usepackage[utf8]{inputenc}
  \usepackage{textcomp} % provide euro and other symbols
\else % if luatex or xetex
  \usepackage{unicode-math}
  \defaultfontfeatures{Scale=MatchLowercase}
  \defaultfontfeatures[\rmfamily]{Ligatures=TeX,Scale=1}
\fi
% Use upquote if available, for straight quotes in verbatim environments
\IfFileExists{upquote.sty}{\usepackage{upquote}}{}
\IfFileExists{microtype.sty}{% use microtype if available
  \usepackage[]{microtype}
  \UseMicrotypeSet[protrusion]{basicmath} % disable protrusion for tt fonts
}{}
\makeatletter
\@ifundefined{KOMAClassName}{% if non-KOMA class
  \IfFileExists{parskip.sty}{%
    \usepackage{parskip}
  }{% else
    \setlength{\parindent}{0pt}
    \setlength{\parskip}{6pt plus 2pt minus 1pt}}
}{% if KOMA class
  \KOMAoptions{parskip=half}}
\makeatother
\usepackage{xcolor}
\IfFileExists{xurl.sty}{\usepackage{xurl}}{} % add URL line breaks if available
\IfFileExists{bookmark.sty}{\usepackage{bookmark}}{\usepackage{hyperref}}
\hypersetup{
  pdftitle={QMIFin HW 1},
  pdfauthor={ArunK},
  hidelinks,
  pdfcreator={LaTeX via pandoc}}
\urlstyle{same} % disable monospaced font for URLs
\usepackage[margin=1in]{geometry}
\usepackage{graphicx,grffile}
\makeatletter
\def\maxwidth{\ifdim\Gin@nat@width>\linewidth\linewidth\else\Gin@nat@width\fi}
\def\maxheight{\ifdim\Gin@nat@height>\textheight\textheight\else\Gin@nat@height\fi}
\makeatother
% Scale images if necessary, so that they will not overflow the page
% margins by default, and it is still possible to overwrite the defaults
% using explicit options in \includegraphics[width, height, ...]{}
\setkeys{Gin}{width=\maxwidth,height=\maxheight,keepaspectratio}
% Set default figure placement to htbp
\makeatletter
\def\fps@figure{htbp}
\makeatother
\setlength{\emergencystretch}{3em} % prevent overfull lines
\providecommand{\tightlist}{%
  \setlength{\itemsep}{0pt}\setlength{\parskip}{0pt}}
\setcounter{secnumdepth}{-\maxdimen} % remove section numbering

\title{QMIFin HW 1}
\author{ArunK}
\date{9/18/2020}

\begin{document}
\maketitle

This homework is partly recap from last year, but has a few open-ended
questions/things we haven't covered :)

\hypertarget{problem-1}{%
\paragraph{Problem 1}\label{problem-1}}

What's the difference between a future and an option? What parameters
are important for pricing an option? What is the classical option
pricing model?

Futures are exchange-traded contracts that specify an underlying to
transact upon at a given price, an exchange date, and relevant
counterparties and holding parties; a future represents an obligation to
deliver or purchase something for an established price at some future
date. Futures are also marked to market daily, resulting in the daily
margin and balance adjustments. An option is a contract between two
parties (can be private) for the right to purchase or sell underlying at
some future time at an established price. The buyer of the option
maintains the option to exercise, while the seller must deliver on their
word if called upon.

Some of the numerous parameters that might contribute to an option's
price: stock price, strike price, option type (includes exercise type,
not really a parameter but I'll count it), time to expiration, interest
rate(s), volatility. If you look at any of the greeks, they're all
quantifications of these parameters (we've only talked about the first
order effects these parameters have in the basic greeks lectures).

The classical option pricing model is the Black Scholes Merton Model:
\[c = S_0\ \text{N}(d_1)-Ke^{-rT}\ \text{N}(d_2) \\ p = Ke^{-rT}\ \text{N}(-d_2)-S_0\ \text{N}(-d_1) \\ \text{where }\  d_1 = \frac{\text{ln}(\frac{S_0}{K})+(r+\frac{\sigma^2}{2})\ T}{\sigma\ \sqrt{T}} \\ d_2 = \frac{\text{ln}(\frac{S_0}{K})+(r-\frac{\sigma^2}{2})\ T}{\sigma\ \sqrt{T}} = d_1 - \sigma T\]

\hypertarget{problem-2}{%
\paragraph{Problem 2}\label{problem-2}}

Using the model identified in P1, price a European Call option on AMGN
with strike price \(\$260\), expiry in exactly 1 year. Pull daily price
info for the last 6 months
\href{https://finance.yahoo.com/quote/AMGN/history?p=AMGN}{here}, and
use the annualized 5-year Treasury rate as the risk free rate
\href{https://www.treasury.gov/resource-center/data-chart-center/interest-rates/pages/textview.aspx?data=yield}{here}.
Please show all work (feel free to check the option chain
\href{https://finance.yahoo.com/quote/AMGN/options/}{here} as a sanity
check)

\emph{Hint: Remember we can price a European call with the BSM model;
the input parameters are time to expiry (needs to match in duration to
the rate used), strike price (given), risk-free rate of return (given,
make sure to use the annualized 5-year rate). The only other things
you'll need are the standard deviation (you should know to use daily
adjusted close prices) calculated with the last 6 months of data, and
the current stock price (you can use open, close, or adjusted close
depending on when you do the homework). What type of standard deviation
should you use?}

\hypertarget{problem-3}{%
\paragraph{Problem 3}\label{problem-3}}

Explain these concepts in 4-5 sentences: CAL, efficient frontier,
minimum variance portfolio, indifference curve, optimal complete
portfolio. (Feel free to include graphics)

\emph{Hint: Remember that these terms all have to do with Modern
Portfolio Theory and the trade-off between portfolio risk and expected
return. How does an investor combine multiple assets into a better
portfolio? What effect does holding varying proportions of risk-free
asset in a portfolio have? Are there certain portfolios that are
strictly better than others?}

\hypertarget{problem-4}{%
\paragraph{Problem 4}\label{problem-4}}

Considering your answers to P1-P3, how well do you feel market behavior
fits our established theoretical framework? Give an empirical example in
support of your answer.

\emph{Hint: Maybe take a look at option pricing chains, factors that
might not be included in the BSM model, variations between MPT and
actual institutional/investor behavior. Data for this can be found
almost anywhere (feel free to use index behavior as a proxy for markets
if that's something you're interested in).}

\hypertarget{problem-5}{%
\paragraph{Problem 5}\label{problem-5}}

What's a KPI you feel has gained importance since February 2020 (any
answer related to Google Trends is not allowed)? Is this KPI
industry-specific or general? Why do you think this KPI was previously
not as important? Ideas to capitalize on this?

\emph{Hint: What company characteristics do investors usually reward? Is
there anything that's changed in consumer behavior or industry
landscapes that make one of these characteristics more valuable? What do
expectations for the next 6-12 months look like? Any actions undertaken
that firms usually try to avoid?}

\hypertarget{problem-6}{%
\paragraph{Problem 6}\label{problem-6}}

Identify and describe specific quantitative factors that can be used to
measure the following: equity liquidity, equity volatility, balance
sheet health, debt quality, short-seller involvement, and momentum.
Please include relevant equations and links -- try to be as specific as
possible and understand there are multiple right answers

\emph{Hint: Think accounting ratios, capital structure, market
participant behavior, and price trends! We're less interested in seeing
niche technical indicators, instead try to develop intuition on ways
investors can quantify financial success and behavior.}

\hypertarget{problem-7}{%
\paragraph{Problem 7}\label{problem-7}}

This past April, WTI Futures settled in Cushing, OK traded at -\$37 per
barrel. Explain what this implied.

\emph{Hint: What does it mean to purchase a future? When is capital
exchanged? When is delivery taken? How do market expectations influence
the ways commodities and their derivatives behave?}

\end{document}
